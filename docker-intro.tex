\PassOptionsToPackage{unicode}{hyperref}
\documentclass[aspectratio=1610]{beamer}

\usetheme[dark]{tudo}

\usepackage[english]{babel}
\usepackage[autostyle]{csquotes}

\usepackage{mathtools}

\usepackage{xcolor}

\usepackage[outputdir=build]{minted}
\usepackage{tcolorbox}
\tcbuselibrary{minted}
\tcbuselibrary{skins}
\tcbuselibrary{xparse}

\usepackage{graphicx}

\usepackage{hyperref}
\usepackage{bookmark}

\unimathsetup{
  math-style=ISO,
  bold-style=ISO,
  nabla=upright,
  partial=upright,
}
\setmathfont{Fira Math}
\setmathfont{XITS Math}[range={cal, bfcal}]
\newfontfamily{\emoji}{Noto Color Emoji}[Renderer=Harfbuzz]

% Codeboxes and input listings
\definecolor{solarizedbg}{HTML}{002B36}
\newcommand\emptyaccsupp[1]{\BeginAccSupp{ActualText={}}#1\EndAccSupp{}}
\renewcommand\theFancyVerbLine{%
  \emptyaccsupp{%
    \footnotesize%
    \textcolor{lightgray}{\arabic{FancyVerbLine}}%
  }%
}
\usemintedstyle{solarized-dark}

\tcbset{basebox/.style={
  listing engine=minted,
  colupper=white,
  colback=solarizedbg,
  colframe=solarizedbg!90!white,
  top=0pt,
  bottom=0pt,
  left=0pt,
  right=0pt,
  enhanced,
}}

\tcbset{codeboxnumbered/.style={
  basebox,
  listing only,
  left=1em,
  overlay={%
    \begin{tcbclipinterior}
      \fill[solarizedbg!80!white] (frame.south west) rectangle ([xshift=1.3em]frame.north west);
    \end{tcbclipinterior}
  },
}}


% Usage: \begin{code}[<tcolorbox options>]<minted options> {<language>} \end{code}
\DeclareTCBListing{codenumbered} {O{} D<>{} m} {
  codeboxnumbered,
  minted language=#3,
  minted options={
    breaklines,
    autogobble,
    linenos,
    numbersep=0.3em,
    stripnl=true,
    #2,
  },
  #1
}

% \codeinputnumbered[<tcolorbox options>]<minted options> {<language>} {path}
\DeclareTCBInputListing{\codeinputnumbered} {O{} D<>{} m m} {
  codeboxnumbered,
  listing file={#4},
  minted language=#3,
  minted options={
    breaklines,
    autogobble,
    linenos,
    numbersep=0.3em,
    stripnl=true,
    #2,
  },
  #1
}

\tcbset{codebox/.style={
  basebox,
  listing only,
}}


% Usage: \begin{code}[<tcolorbox options>]<minted options> {<language>} \end{code}
\DeclareTCBListing{code} {O{} D<>{} m} {
  codebox,
  minted language=#3,
  minted options={
    breaklines,
    autogobble,
    stripnl=true,
    #2,
  },
  #1
}

% \codeinput[<tcolorbox options>]<minted options> {<language>} {path}
\DeclareTCBInputListing{\codeinput} {O{} D<>{} m m} {
  codebox,
  listing file={#4},
  minted language=#3,
  minted options={
    breaklines,
    autogobble,
    stripnl=true,
    #2,
  },
  #1
}



\title{A short Introduction to Docker / Containers}%

\author[M. Linhoff]{Maximilian Linhoff}
\date[2024-04-03]{APP Group Seminar 2024-04-03}
\institute[{%
  \includegraphics[height=\headerheight]{./applogo.pdf}%
}]{Astroparticle Physics | TU Dortmund University}
\titlegraphic{{\includegraphics[height=0.25\textheight]{./build/docker-logo.pdf}}}


\begin{document}
\maketitle
\tableofcontents

\section{What are Containers?}
\begin{frame}{What are Containers?}
  \begin{itemize}
    \item Containers are a more lightweight solution to get isolated, reproducible environments for software compared to full Virtual Machines
    \item Using different Linux features to isolate:
      \begin{itemize}
        \item Files
        \item Resources (CPU, Memory, Disk space, ...)
        \item Network
      \end{itemize}
    \item On Windows and MacOS, a Linux VM will be used to run the Linux containers inside
  \end{itemize}
\end{frame}

\begin{frame}{Usecases}
  \begin{itemize}
    \item Getting an isolated, reproducible software environment that might be different from the host system
      \begin{itemize}
        \item Running legacy software on a current OS
        \item Running current software on a legacy OS (if its new enough to run docker...)
        \item Saving a software environment for reproducibility
      \end{itemize}
    \item Easily setup services like databases, webservers, etc.
    \item Create isolated setups of services, like a GitLab that needs a webserver, a database, a CI runner, ...
    \item ...
  \end{itemize}
\end{frame}

\begin{frame}{Docker}
  \begin{itemize}
    \item Docker is one implementation of such a container system
    \item But container formats and tooling is now mostly standardized: Open Container Initative (OCI)
    \item Basic parts of Docker are open source, but e.g. Docker Desktop is not
    \item Alternatives: podman, Singularity / Apptainer, ...
  \end{itemize}
\end{frame}


\begin{frame}{Some Terminology}
  \begin{description}
    \item[Image] A read-only template for creating a new Container. \\
      Identified by a registry, namespace, name and version tag \\
      E.g.: \texttt{docker.io/library/ubuntu:22.04}, the offical docker image for Ubuntu 22.04 \\
      Can be shorted to \texttt{ubuntu:22.04} or even \texttt{ubuntu} \\ $⇒$ defaults for registry (docker.io), namespace (library) and tag (latest)
    \item[Container] An \emph{Instance} of an image. Isolated from the rest of the system. \\
      Can have resource constraints, networking, volumes etc. configured.
    \item[Registry] A server hosting docker images
  \end{description}
\end{frame}

\begin{frame}{Security}
  \begin{itemize}
    \item Being able to start and run docker containers has security implications
    \item You can essentially become root through starting a docker container and mounting directories of the host
    \item[$⇒$] Many environments don't allow starting docker containers for normal users (Clusters, CI, ...) \\
      Apptainer is a tool that works a bit differently, doesn't have this issue and can run docker containers
  \end{itemize}
\end{frame}

\section{Using Docker}
\begin{frame}[fragile]{Starting your first container}
  \begin{codenumbered}{text}
    $ docker run --rm -it ubuntu:22.04 bash
  \end{codenumbered}

  \begin{description}
    \item[\texttt{--rm}] Remove container after logging out 
    \item[\texttt{-it}] open an interactive tty shell
  \end{description}
\end{frame}

\begin{frame}[fragile]{Volumes}
  \begin{itemize}
    \item Bind-mounting paths from the host into the container:
      \begin{codenumbered}{text}
        $ docker run --rm \
          -v /path/on/host:/path/in/container \
          ubuntu:22.04 \
          ls /path/in/container
      \end{codenumbered}
    \item Using a docker volume, data is stored somewhere in docker's internal area on the host:
      \begin{codenumbered}{text}
        $ docker volume create foo
        $ docker run --rm -v
          -v foo:/path/in/container \
          ubuntu:22.04 \
          ls /path/in/container
      \end{codenumbered}
  \end{itemize}
\end{frame}

\begin{frame}[fragile]{Ports, running stuff in the background}
      \begin{codenumbered}{text}
        $ docker run -d \
          -p 5432:5432 \
          -e POSTGRES_USER=foo \
          -e POSTGRES_PASSWORD=top-secret \
          postgres
      \end{codenumbered}

      \begin{description}
        \item[-d] Run in the background
        \item[-p host:container] Map a port from host to container 
        \item[-e key=value] Set environment variables inside the container 
      \end{description}
\end{frame}

\begin{frame}[fragile]{Creating your own images: Dockerfile}
  Docker images are described in text recipes to \emph{built} an image:
  \codeinputnumbered[title={\texttt{examples/simple/Dockerfile}}]{docker}{./examples/simple/Dockerfile}
\end{frame}

\begin{frame}[fragile]{Creating your own images: Dockerfile}
  \begin{code}{text}
    $ cd examples/simple
    $ docker build -t example-simple .
    ... lots of output
    $ docker run --rm example-simple
    Hello, Docker!
  \end{code}
\end{frame}

\section{Docker Compose}
\begin{frame}{What if you have multiple things that need to work together?}
\end{frame}
\section{Docker and Security}
\section{Singularity / Apptainer}

\end{document}
