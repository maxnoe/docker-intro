\PassOptionsToPackage{unicode}{hyperref}
\documentclass[aspectratio=1610]{beamer}

\usetheme[dark]{tudo}

\usepackage[english]{babel}
\usepackage[autostyle]{csquotes}

\usepackage{mathtools}

\usepackage{xcolor}

\usepackage[outputdir=build]{minted}
\usepackage{tcolorbox}
\tcbuselibrary{minted}
\tcbuselibrary{skins}
\tcbuselibrary{xparse}

\usepackage{tabularray}

\usepackage{graphicx}

\usepackage{accsupp}
\usepackage{hyperref}
\usepackage{bookmark}

\unimathsetup{
  math-style=ISO,
  bold-style=ISO,
  nabla=upright,
  partial=upright,
}
\setmathfont{Fira Math}
\setmathfont{XITS Math}[range={cal, bfcal}]
\newfontfamily{\emoji}{Noto Color Emoji}[Renderer=Harfbuzz]

% Codeboxes and input listings
\definecolor{solarizedbg}{HTML}{002B36}
\newcommand\emptyaccsupp[1]{\BeginAccSupp{ActualText={}}#1\EndAccSupp{}}
\renewcommand\theFancyVerbLine{%
  \emptyaccsupp{%
    \footnotesize%
    \textcolor{lightgray}{\arabic{FancyVerbLine}}%
  }%
}
\usemintedstyle{solarized-dark}

\tcbset{basebox/.style={
  listing engine=minted,
  colupper=white,
  colback=solarizedbg,
  colframe=solarizedbg!90!white,
  top=0pt,
  bottom=0pt,
  left=0pt,
  right=0pt,
  enhanced,
}}

\tcbset{codeboxnumbered/.style={
  basebox,
  listing only,
  left=1em,
  overlay={%
    \begin{tcbclipinterior}
      \fill[solarizedbg!80!white] (frame.south west) rectangle ([xshift=1.3em]frame.north west);
    \end{tcbclipinterior}
  },
}}


% Usage: \begin{code}[<tcolorbox options>]<minted options> {<language>} \end{code}
\DeclareTCBListing{codenumbered} {O{} D<>{} m} {
  codeboxnumbered,
  minted language=#3,
  minted options={
    breaklines,
    autogobble,
    linenos,
    numbersep=0.3em,
    stripnl=true,
    #2,
  },
  #1
}

% \codeinputnumbered[<tcolorbox options>]<minted options> {<language>} {path}
\DeclareTCBInputListing{\codeinputnumbered} {O{} D<>{} m m} {
  codeboxnumbered,
  listing file={#4},
  minted language=#3,
  minted options={
    breaklines,
    autogobble,
    linenos,
    numbersep=0.3em,
    stripnl=true,
    #2,
  },
  #1
}

\tcbset{codebox/.style={
  basebox,
  listing only,
}}


% Usage: \begin{code}[<tcolorbox options>]<minted options> {<language>} \end{code}
\DeclareTCBListing{code} {O{} D<>{} m} {
  codebox,
  minted language=#3,
  minted options={
    breaklines,
    autogobble,
    stripnl=true,
    #2,
  },
  #1
}

% \codeinput[<tcolorbox options>]<minted options> {<language>} {path}
\DeclareTCBInputListing{\codeinput} {O{} D<>{} m m} {
  codebox,
  listing file={#4},
  minted language=#3,
  minted options={
    breaklines,
    autogobble,
    stripnl=true,
    #2,
  },
  #1
}



\title{A short Introduction to Docker / Containers}%

\author[M. Linhoff]{Maximilian Linhoff}
\date[2024-04-03]{APP Group Seminar 2024-04-03}
\institute[{%
  \includegraphics[height=\headerheight]{./applogo.pdf}%
}]{Astroparticle Physics | TU Dortmund University}
\titlegraphic{{\includegraphics[height=0.25\textheight]{./build/docker-logo.pdf}}}


\begin{document}
\maketitle
\tableofcontents

\section{What are Containers?}
\begin{frame}{What are Containers?}
  \begin{itemize}
    \item Containers are a more lightweight solution to get isolated, reproducible environments for software compared to full Virtual Machines
    \item Using different Linux features to isolate:
      \begin{itemize}
        \item Files
        \item Resources (CPU, Memory, Disk space, ...)
        \item Network
      \end{itemize}
    \item On Windows and MacOS, a Linux VM will be used to run the Linux containers inside
  \end{itemize}
\end{frame}

\begin{frame}{Usecases}
  \begin{itemize}
    \item Getting an isolated, reproducible software environment that might be different from the host system
      \begin{itemize}
        \item Running legacy software on a current OS
        \item Running current software on a legacy OS (if its new enough to run docker...)
        \item Saving a software environment for reproducibility
      \end{itemize}
    \item Easily setup services like databases, webservers, etc.
    \item Create isolated setups of services, like a GitLab that needs a webserver, a database, a CI runner, ...
    \item ...
  \end{itemize}
\end{frame}

\begin{frame}{Docker}
  \begin{itemize}
    \item Docker is one implementation of such a container system
    \item But container formats and tooling is now mostly standardized: Open Container Initative (OCI)
    \item Basic parts of Docker are open source, but e.g. Docker Desktop is not
    \item Alternatives: podman, Singularity / Apptainer, ...
  \end{itemize}
\end{frame}


\begin{frame}{Some Terminology}
  \begin{description}
    \item[Image] A read-only template for creating a new Container. \\
      Identified by a registry, namespace, name and version tag \\
      E.g.: \texttt{docker.io/library/ubuntu:22.04}, the offical docker image for Ubuntu 22.04 \\
      Can be shorted to \texttt{ubuntu:22.04} or even \texttt{ubuntu} \\ $⇒$ defaults for registry (docker.io), namespace (library) and tag (latest)
    \item[Container] An \emph{Instance} of an image. Isolated from the rest of the system. \\
      Can have resource constraints, networking, volumes etc. configured.
    \item[Registry] A server hosting docker images
  \end{description}
\end{frame}

\begin{frame}{Security}
  \begin{itemize}
    \item Being able to start and run docker containers has security implications
    \item You can essentially become root through starting a docker container and mounting directories of the host
    \item[$⇒$] Many environments don't allow starting docker containers for normal users (Clusters, CI, ...) \\
      Apptainer is a tool that works a bit differently, doesn't have this issue and can run docker containers
  \end{itemize}
\end{frame}

\section{Using Docker}
\begin{frame}[fragile]{Starting your first container}
  \begin{code}{text}
    $ docker run --rm -it ubuntu:22.04 bash
  \end{code}

  \begin{description}
    \item[\texttt{--rm}] Remove container after logging out 
    \item[\texttt{-it}] open an interactive tty shell
  \end{description}
\end{frame}

\begin{frame}[fragile]{Ports, running stuff in the background}
      \begin{code}{text}
        $ docker run -d \
          -p 5432:5432 \
          -e POSTGRES_USER=foo \
          -e POSTGRES_PASSWORD=top-secret \
          --name awesome-db \
          postgres
      \end{code}

      \begin{description}
        \item[-d] Run in the background
        \item[-p host:container] Map a port from host to container 
        \item[-e key=value] Set environment variables inside the container 
      \end{description}
\end{frame}

\begin{frame}[fragile]{See what is currently running}

  {\tiny
  \begin{code}{text}
    $ docker ps
    CONTAINER ID IMAGE    COMMAND                 CREATED        STATUS       PORTS                                     NAMES
    71c4cb81a3ff postgres "docker-entrypoint.s…"  2 seconds ago  Up 1 second  0.0.0.0:5432->5432/tcp, :::5432->5432/tcp awesome-db
  \end{code}
  }

  \bigskip
  Containers can be stopped, restarted, deleted
  {\tiny
  \begin{code}{text}
    $ docker stop awesome-db
    $ docker ps
    $ docker ps -a
    CONTAINER ID IMAGE    COMMAND                 CREATED            STATUS                    NAMES
    71c4cb81a3ff postgres "docker-entrypoint.s…"  About a minute ago Exited (0) 2 seconds ago  awesome-db
    $ docker start awesome-db
    $ docker stop awesome-db
    $ docker rm awesome-db
  \end{code}
  }
\end{frame}

\begin{frame}[fragile]{Volumes}
  \begin{itemize}
    \item Containers need access to persistent storage
    \item Database content should not be lost when a container is removed\\
      this happens e.\,g.\ when updating a service to a new version of the image
    \item To persist data, you need to know the path inside the container where the
      important data is stored, some examples:
      \begin{description}
        \item[MySQL] \mintinline{text}+/var/lib/mysql+
        \item[PostgreSQL] \mintinline{text}+/var/lib/postgresql/data+
        \item[Nextcloud] \mintinline{text}+/var/www/html/data+
      \end{description}
    \item docs: \url{https://docs.docker.com/storage/volumes/}
  \end{itemize}
\end{frame}

\begin{frame}[fragile]{Bind-mounting Volumes}
  \begin{itemize}
    \item Bind-mounting paths from the host into the container:
      \begin{code}{text}
        $ docker run --rm \
          -v /path/on/host:/path/in/container \
          ubuntu:22.04 \
          ls /path/in/container
      \end{code}
    \item Can be used to persist data from the container to the host
    \item But also to access data from the host, more relevant for e.\,g.\ analysis software in docker containers
    \item You can pass options, e.\,g.\ making the mount read-only: \mintinline{text}+-v /path/on/host:/path/in/container:ro+
  \end{itemize}
\end{frame}

\begin{frame}[fragile]{Docker volumes Volumes}
  \begin{itemize}
    \item Using a docker volume, data is stored somewhere in docker's internal area on the host:
      \begin{code}{text}
        $ docker volume create awesome-db-data
        $ docker run -d \
          -p 5432:5432 \
          -e POSTGRES_USER=foo \
          -e POSTGRES_PASSWORD=top-secret \
          -v awesome-db-data:/var/lib/postgresql/data \
          --name awesome-db \
          postgres
      \end{code}
  \end{itemize}
\end{frame}

\begin{frame}[fragile]{Creating your own images: Dockerfile}
  Docker images are described in text recipes to \emph{built} an image:
  \codeinputnumbered[title={\texttt{examples/simple/Dockerfile}}]{docker}{./examples/simple/Dockerfile}
\end{frame}

\begin{frame}[fragile]{Creating your own images: Dockerfile}
  \begin{code}{text}
    $ cd examples/simple
    $ docker build -t example-simple .
    ... lots of output
    $ docker run --rm example-simple
    Hello, Docker!
  \end{code}
\end{frame}

\begin{frame}[fragile]{Creating your own images: Dockerfile}
  \begin{itemize}
    \item Docker images are made up of immutable layers
    \item Each command in the Dockerfile will create a new layer
    \item Files created in one layer and deleted in the next will still increase the image size
    \item Avoid creating temporary files that persist in a layer, some examples:
      \begin{itemize}
        \item pip: \mintinline{text}+pip install --no-cache numpy+
        \item apt packages
          \begin{code}{text}
            apt-get update \
            && apt-get install -y git \
            && rm -rf /var/lib/apt/lists/* 
          \end{code}
        \item for more see: \href{https://docs.docker.com/develop/develop-images/dockerfile_best-practices/}{Dockerfile best practices}
      \end{itemize}
  \end{itemize}
\end{frame}

\begin{frame}[fragile]{Creating your own images: Dockerfile}{}

  {\small
  \begin{code}[title={Good!}]{docker}
    RUN curl -L https://root.cern/download/root_v6.30.06.source.tar.gz | tar xz \
      && cmake -S root-6.30.06 -B root-build [...] \
      && [...] \
      && rm -rf root-build root-6.30.06
  \end{code}
  }

  {\small
    \begin{code}[title={Very bad, will result in large image size!}]{docker}
      RUN curl -LO https://root.cern/download/root_v6.30.06.source.tar.gz
      RUN tar xzf root_v6.30.06.source.tar.gz
      WORKDIR /opt/root-6.30.06 
      RUN cmake -S root-6.30.06 -B root-build [...]
      RUN [...]
      RUN rm -r /opt/root-6.30.06
      RUN rm /root_v6.30.06.sources.tar.gz
    \end{code}
  }
\end{frame}

\begin{frame}{Some useful commands}
  \begin{tblr}{
      colspec = {l X},
      column{1} = {font=\ttfamily},
    }
    docker ps           & List running containers \\
    docker ps -a        & List all, also stopped containers \\
    docker exec -it <container> <bash> &  Run a command in a running container \\
    docker images       & List images \\
    docker image prune  & Removes \emph{dangling} images. Can free lots of space. \\
    docker system prune & Removes everything not currently in use \newline (containers, images, volumes, ...). \\
  \end{tblr}
\end{frame}

\section{Docker Compose}
\begin{frame}{What if you have multiple things that need to work together?}
  Docker compose allows describing multiple containers in yaml file
  \begin{itemize}
    \item Basically all features of the docker command line supported
    \item Resources
    \item Volumes
    \item Networks
    \item Ports
    \item ...
  \end{itemize}
\end{frame}

\begin{frame}{Basic docker compose example: Wiki.js}
  \footnotesize%
  \only<2>{
    \codeinputnumbered[title={./examples/compose/docker-compose.yaml}]<firstline=1, lastline=12>{yaml}{./examples/compose/docker-compose.yaml}
  }%
  \only<3>{
    \codeinputnumbered[title={./examples/compose/docker-compose.yaml}]<firstline=14, lastline=28>{yaml}{./examples/compose/docker-compose.yaml}
  }%
  \only<4>{
    \codeinputnumbered[title={./examples/compose/docker-compose.yaml}]<firstline=30, lastline=34>{yaml}{./examples/compose/docker-compose.yaml}
  }%
\end{frame}

\begin{frame}{}
\end{frame}

\section{Apptainer / Singularity}
\begin{frame}[fragile]{Apptainer / Singularity}
  \begin{itemize}
    \item Apptainer (formerly Singularity) is a container engine that solves some of the security issues around docker in shared environments
    \item Main feature: containers are run in user namespaces, not by a root-owned service
    \item By default: access to current pwd and home
    \item Uses different, custom container format: SIF
    \item But can automatically convert docker images to SIF, e.g.:
  \end{itemize}
\end{frame}

\begin{frame}[fragile]{Apptainer / Singularity}
  \footnotesize
  \begin{code}{text}
    $ id
    uid=5074(mnoethe) gid=5002(app) groups=5002(app),120(lpadmin),998(docker),5000(e5),5012(app_admin), 5014(cta),5015(magic),5017(fact),10024(lidong-uph),11000(exp),11200(hep)
    $ pwd
    /home/mnoethe/Uni/e5b/apptainer-demo
    $ head -n1 /etc/os-release
    PRETTY_NAME="Ubuntu 22.04.4 LTS"

    $ apptainer run docker://almalinux:9
    Apptainer> id
    uid=5074(mnoethe) gid=5002(app) groups=5002(app),65534(nobody)
    Apptainer> pwd
    /home/mnoethe/Uni/e5b/apptainer-demo
    Apptainer> head -n1 /etc/os-release 
    NAME="AlmaLinux"
  \end{code}
\end{frame}

\end{document}
